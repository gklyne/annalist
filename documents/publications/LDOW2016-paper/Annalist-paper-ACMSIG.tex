% This is "sig-alternate.tex" V2.1 April 2013
% This file should be compiled with V2.5 of "sig-alternate.cls" May 2012
%
% This example file demonstrates the use of the 'sig-alternate.cls'
% V2.5 LaTeX2e document class file. It is for those submitting
% articles to ACM Conference Proceedings WHO DO NOT WISH TO
% STRICTLY ADHERE TO THE SIGS (PUBS-BOARD-ENDORSED) STYLE.
% The 'sig-alternate.cls' file will produce a similar-looking,
% albeit, 'tighter' paper resulting in, invariably, fewer pages.
%
% ----------------------------------------------------------------------------------------------------------------
% This .tex file (and associated .cls V2.5) produces:
%       1) The Permission Statement
%       2) The Conference (location) Info information
%       3) The Copyright Line with ACM data
%       4) NO page numbers
%
% as against the acm_proc_article-sp.cls file which
% DOES NOT produce 1) thru' 3) above.
%
% Using 'sig-alternate.cls' you have control, however, from within
% the source .tex file, over both the CopyrightYear
% (defaulted to 200X) and the ACM Copyright Data
% (defaulted to X-XXXXX-XX-X/XX/XX).
% e.g.
% \CopyrightYear{2007} will cause 2007 to appear in the copyright line.
% \crdata{0-12345-67-8/90/12} will cause 0-12345-67-8/90/12 to appear in the copyright line.
%
% ---------------------------------------------------------------------------------------------------------------
% This .tex source is an example which *does* use
% the .bib file (from which the .bbl file % is produced).
% REMEMBER HOWEVER: After having produced the .bbl file,
% and prior to final submission, you *NEED* to 'insert'
% your .bbl file into your source .tex file so as to provide
% ONE 'self-contained' source file.
%
% ================= IF YOU HAVE QUESTIONS =======================
% Questions regarding the SIGS styles, SIGS policies and
% procedures, Conferences etc. should be sent to
% Adrienne Griscti (griscti@acm.org)
%
% Technical questions _only_ to
% Gerald Murray (murray@hq.acm.org)
% ===============================================================
%
% For tracking purposes - this is V2.0 - May 2012

\documentclass{sig-alternate-05-2015}

\usepackage{url}
\usepackage[breaklinks]{hyperref}
% \usepackage{breakurl} 
% \Urlmuskip=0mu plus 1mu\relax

% See:
%   http://stackoverflow.com/questions/3175105/how-to-insert-code-into-a-latex-doc 
%   http://tex.stackexchange.com/questions/83085/how-to-improve-listings-display-of-json-files
\usepackage{listings}
\usepackage{color}
\usepackage{microtype}
\usepackage{needspace}

% \setlength{\belowcaptionskip}{-40pt}
% \setlength{\abovecaptionskip}{-10pt}

\definecolor{dkgreen}{rgb}{0,0.6,0}
\definecolor{gray}{rgb}{0.5,0.5,0.5}
\definecolor{mauve}{rgb}{0.58,0,0.82}
%\definecolor{background}{HTML}{EEEEEE}
\definecolor{delim}{RGB}{20,105,176}

% \colorlet{punct}{red!60!black}
% \colorlet{numb}{magenta!60!black}

% \lstdefinelanguage{json}{
%     basicstyle=\small\ttfamily,
%     %numbers=left,
%     %numberstyle=\scriptsize,
%     %stepnumber=1,
%     %numbersep=8pt,
%     showstringspaces=false,
%     breaklines=true,
%     frame=lines,
%     backgroundcolor=\color{background},
% }

\lstset{frame=tb,
  language=Java,
  aboveskip=3mm,
  belowskip=3mm,
  showstringspaces=false,
  columns=flexible,
  basicstyle={\small\ttfamily},
  numbers=none,
  numberstyle=\tiny\color{gray},
  keywordstyle=\color{blue},
  commentstyle=\color{mauve},
  stringstyle=\color{dkgreen},
  breaklines=true,
  breakatwhitespace=true,
  tabsize=3
}

\usepackage[backend=bibtex, style=trad-abbrv]{biblatex}
\bibliography{Annalist-refs}
% \setcounter{biburlnumpenalty}{100}

\providecommand{\tightlist}{%
  \setlength{\itemsep}{0pt}\setlength{\parskip}{0pt}}

\begin{document}

% Copyright
\setcopyright{none}
%\setcopyright{acmcopyright}
%\setcopyright{acmlicensed}
%\setcopyright{rightsretained}
%\setcopyright{usgov}
%\setcopyright{usgovmixed}
%\setcopyright{cagov}
%\setcopyright{cagovmixed}

% DOI
% \doi{10.475/123_4}

% ISBN
% \isbn{123-4567-24-567/08/06}

%Conference
\conferenceinfo{LDOW2016}{April 11--15, 2016, Montreal, Canada}

%\acmPrice{\$15.00}

%
% --- Author Metadata here ---
%\conferenceinfo{WOODSTOCK}{'97 El Paso, Texas USA}
%\CopyrightYear{2007} % Allows default copyright year (20XX) to be over-ridden - IF NEED BE.
%\crdata{0-12345-67-8/90/01}  % Allows default copyright data (0-89791-88-6/97/05) to be over-ridden - IF NEED BE.
% --- End of Author Metadata ---

\title{
  Annalist
  % \titlenote{(@@titlenote...@@)}
  }
\subtitle{A practical tool for creating, managing and sharing evolving linked data}

% \titlenote{A full version of this paper is available as
% \textit{Author's Guide to Preparing ACM SIG Proceedings Using
% \LaTeX$2_\epsilon$\ and BibTeX} at
% \texttt{www.acm.org/eaddress.htm}}}

%
% You need the command \numberofauthors to handle the 'placement
% and alignment' of the authors beneath the title.
%
% For aesthetic reasons, we recommend 'three authors at a time'
% i.e. three 'name/affiliation blocks' be placed beneath the title.
%
% NOTE: You are NOT restricted in how many 'rows' of
% "name/affiliations" may appear. We just ask that you restrict
% the number of 'columns' to three.
%
% Because of the available 'opening page real-estate'
% we ask you to refrain from putting more than six authors
% (two rows with three columns) beneath the article title.
% More than six makes the first-page appear very cluttered indeed.
%
% Use the \alignauthor commands to handle the names
% and affiliations for an 'aesthetic maximum' of six authors.
% Add names, affiliations, addresses for
% the seventh etc. author(s) as the argument for the
% \additionalauthors command.
% These 'additional authors' will be output/set for you
% without further effort on your part as the last section in
% the body of your article BEFORE References or any Appendices.

\numberofauthors{3}
\author{
% NOTE: blank lines here cause problems
%
% You can go ahead and credit any number of authors here,
% e.g. one 'row of three' or two rows (consisting of one row of three
% and a second row of one, two or three).
%
% The command \alignauthor (no curly braces needed) should
% precede each author name, affiliation/snail-mail address and
% e-mail address. Additionally, tag each line of
% affiliation/address with \affaddr, and tag the
% e-mail address with \email.
%
% 1st. author
\alignauthor
Graham Klyne \titlenote{Corresponding author}\\
       \affaddr{Oxford e-Research Centre}\\
       \affaddr{University of Oxford}\\
       \affaddr{7 Keble Rd}\\
       \affaddr{Oxford, OX1 3QG, UK}\\
       \email{graham.klyne@oerc.ox.ac.uk}
% 2nd. author
\alignauthor
Cerys Willoughby \\ %\titlenote{}\\
       \affaddr{Chemistry}\\
       \affaddr{University of Southampton}\\
       \affaddr{Highfield}\\
       \affaddr{Southampton, SO17 1BJ, UK}\\
       \email{Cerys.Willoughby@soton.ac.uk}
% 3rd. author
\alignauthor
Kevin Page \\ %\titlenote{}\\
       \affaddr{Oxford e-Research Centre}\\
       \affaddr{University of Oxford}\\
       \affaddr{7 Keble Rd}\\
       \affaddr{Oxford, OX1 3QG, UK}\\
       \email{kevin.page@oerc.ox.ac.uk}
}

% There's nothing stopping you putting the seventh, eighth, etc.
% author on the opening page (as the 'third row') but we ask,
% for aesthetic reasons that you place these 'additional authors'
% in the \additional authors block, viz.
% \additionalauthors{Additional authors: John Smith (The Th{\o}rv{\"a}ld Group,
% email: {\texttt{jsmith@affiliation.org}}) and Julius P.~Kumquat
% (The Kumquat Consortium, email: {\texttt{jpkumquat@consortium.net}}).}
% \date{30 July 1999}
% Just remember to make sure that the TOTAL number of authors
% is the number that will appear on the first page PLUS the
% number that will appear in the \additionalauthors section.


% Suppress space reserved for ACM copyright...
% See: http://tex.stackexchange.com/questions/21536/
\makeatletter
\def\ftype@copyrightbox{8}
\def\@copyrightspace{
\@float{copyrightbox}[b]\crnotice{
Copyright is held by the author/owner(s).

WWW2016 Workshop: Linked Data on the Web (LDOW2016)}
\end@float
}
\makeatother

\maketitle

\begin{abstract}

% Annalist is a software system for individuals and small groups to reap the benefits of using RDF linked data, creating data that participates in a wider web of linked data.  It presents a flexible web interface for creating, editing and browsing evolvable data, without requiring the user to be familiar with minutiae of the RDF model or syntax, or to perform any programming, HTML coding or prior configuration.

Annalist is a software system for individuals and small groups to reap the benefits of using RDF linked data, supporting them to easily create data that participates in a wider web of linked data. It presents a flexible web interface for creating, editing and browsing evolvable data, without requiring the user to be familiar with minutiae of the RDF model or syntax, or to perform any programming, HTML coding or prior configuration.

Development of Annalist was motivated by data capture and sharing concerns in a small bioinformatics research group, and customized personal information management.  Requirements centre particularly on achieving low activation energy for simple tasks, flexibility to add structural details as data is collected, access-controlled sharing, and ability to connect private data with public data on the web.  It is designed as a web server application, presenting an interface for defining data structure and managing data.  Data is stored as text files that are amenable to access by existing software, with the intent that a range of applications may be used in concert to gather, manage and publish data.

During its development, Annalist has been used in a range of applications, which have informed decisions about its design and proven its flexibility and robustness in use. It has been particularly effective in exploring and rapid prototyping designs for linked data on the web, covering science and humanities research, creative art and personal information.

\end{abstract}


%
% The code below should be generated by the tool at
% http://dl.acm.org/ccs.cfm
% Please copy and paste the code instead of the example below. 
%

\begin{CCSXML}
<ccs2012>
<concept>
<concept_id>10002951.10003260.10003309.10003315.10003314</concept_id>
<concept_desc>Information systems~Resource Description Framework (RDF)</concept_desc>
<concept_significance>500</concept_significance>
</concept>
<concept>
<concept_id>10002951.10002952</concept_id>
<concept_desc>Information systems~Data management systems</concept_desc>
<concept_significance>300</concept_significance>
</concept>
<concept>
<concept_id>10010405.10010444.10010450</concept_id>
<concept_desc>Applied computing~Bioinformatics</concept_desc>
<concept_significance>300</concept_significance>
</concept>
<concept>
<concept_id>10010405.10010432.10010436</concept_id>
<concept_desc>Applied computing~Chemistry</concept_desc>
<concept_significance>300</concept_significance>
</concept>
<concept>
<concept_id>10010405.10010469</concept_id>
<concept_desc>Applied computing~Arts and humanities</concept_desc>
<concept_significance>300</concept_significance>
</concept>
</ccs2012>
\end{CCSXML}

\ccsdesc[500]{Information systems~Resource Description Framework (RDF)}
\ccsdesc[300]{Information systems~Data management systems}
\ccsdesc[300]{Applied computing~Bioinformatics}
\ccsdesc[300]{Applied computing~Chemistry}
\ccsdesc[300]{Applied computing~Arts and humanities}

%
% End generated code
%

%
%  Use this command to print the description
%
\printccsdesc

% We no longer use \terms command
%\terms{Theory}

\keywords{Semantic Web, Linked data, Data management}

\input{Annalist-paper-body}

%
% The following two commands are all you need in the
% initial runs of your .tex file to
% produce the bibliography for the citations in your paper.

\printbibliography

% \bibliographystyle{abbrv}
% \bibliography{sigproc}  % sigproc.bib is the name of the Bibliography in this case

% You must have a proper ".bib" file
%  and remember to run:
% latex bibtex latex latex
% to resolve all references
%
% ACM needs 'a single self-contained file'!
%

\end{document}
